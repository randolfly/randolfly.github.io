\section{引言}

500米口径大型球面射电望远镜(Five hundred meters Aperture Spherical Telescope, FAST项目)是中国自主设计、建造完成的世界上现存最大的,
灵敏度最高的单口径球面射电望远镜,也是国家的九大科技基础设施之一。FAST主体结构修建在贵州独特的喀斯特地貌上,主要包含有主动反射面和馈源支撑系统两个部分,
其中馈源支撑系统上搭载着信号接收装置,需要完成指定的位姿调整才能实现精准的信号接收。馈源支撑系统是一个刚柔耦合二级机构,
由一个6自由度Stewart平台级联在一个6索驱动并联装置上构成\cite{pengFiveHundredMeterApertureSpherical2001}\cite{nanFiveHundredMeter2006}。
在结合力反馈的情况下,FAST位置控制精度可以使得其完成70 MHz到3 GHz的信号观测,而进一步提升馈源支撑系统的位置控制精度可以提升其观测精度,这正是本文的控制目标。

从实验环境的特点来看,由于馈源支撑系统整体悬吊在空中,其所受的外力仅仅只有风扰。根据实地观测结果和对风扰的研究,其外界扰动低频有界。
从实验装置的特点来看,这样的结构成功满足了理论上的运动条件需求,但其自身的动力学特性却带来了其他的控制问题。绳索的弹性特点使得整个馈源舱的系统刚度较低,
很容易受到外界干扰;Stewart平台相对绳索自重不可忽略,其运动过程产生的反作用力会对其自身的运动产生影响,从而使得机构呈现出一定的耦合特性。

针对FAST馈源支撑机构主动抑振问题的研究主要有两个思路,一是进行完整的系统建模,通过模型来完成前馈控制。如刘志华在2015年提出了针对刚柔混联机构的
主动抑振措施,将Ste\\wart平台支链运动也放入模型考虑。在使用绳索弹簧-阻尼模型情况下,将平台的伸缩支链的驱动力视为内力,建立了系统整体的动力学模型,
从而计算得到理论情况伸缩支链运动所需要的动力学耦合补偿项\cite{liuGangRouHunLianJiGouDeZhenDongTeXingJiYiZhenKongZhiYanJiu2015}。
另一个思路则是使用类似自适应控制等的无模型的控制方法。段学超在2010年针对FAST的50m缩尺模型提出了一种自适应交互PID监督控制器。
作者首先通过二阶微分跟踪器在支链长度噪声信号中提取支链速度、加速度信息,然后使用自适应交互算法实时更新PID参数,
实现了Stewart平台的监督控制\cite{duanRouXingZhiChengStewartPingTaiDeFenXiYouHuaYuKongZhiYanJiu2008}。

本文的控制对象属于宏-微系统,对这类系统,其振动特性与刚性的并联机构有所不同,主要区别有两方面:一方面是由于底部的宏观柔性支撑系统自身刚性差,
容易受到外部或者系统内部因素的影响而发生振动;另一方面,柔性支撑系统与其上的刚性并联机构之间存在动力耦合现象,即底部的柔性支撑系统在运动中,
给其基础上的刚性平台带来了科里奥利力等由于非惯性系而出现的力,而上面的刚性平台在运动过程中的反作用力也会对底部支撑系统产生影响,
从而使整个系统的动力学特性变得非常复杂。

对刚柔混联的宏-微系统,学者们主要提出了两种抑振策略:主动抑振策略和被动抑振策略。

被动抑振指的是系统对大的振动不做主动干预,利用系统存在的摩擦力等非保守力和阻尼等因素,使得振动的能量耗散掉,直到振动耗散到可以接受的范围后再进行操作。
主动抑振指的是通过一定的策略,根据观测到的系统状态,对执行器进行连续的控制操作,将目标的振动幅值、频率等控制在可接受的范围。

一般而言,被动抑振适用于精度不太高的情形,研究的学者比较少。对主动抑振策略,有直接通过研究补偿量的方法。从优化角度来看,Yoshikawa使用补偿量最小的原则,
在多自由度的微机器人上进行控制,使基座产生的弹性变形得到补偿,完成了系统的轨迹规划控制\cite{yoshikawaQuasistaticTrajectoryTracking1993}。
从模型角度来看,Parsa利用计算力矩法,对冗余自由度的柔性支撑机器人进行了动力学分析,利用广义逆矩阵方法,使上部刚性机器人产生的振动直接抵消柔性支撑的振动,
并在一个平面的柔性支撑机构上进行了算法验证\cite{parsaControlMacroMicroManipulators2005}。

另一方面,从能量角度来进行宏-微系统的主动抑振。这一思想主要是使振动能量随着时间不断耗散,从而实现抑振。其思想是继承于被动抑振的,
但其可以主动调节控制器来调控能量耗散的情况。Torres基于一定的假设,简化并建立了系统的动力学模型,提出伪被动能量耗散方法,本质是将机器人的特性逼近于一个阻尼器,
同时调节控制器的增益来实现大的能量耗散\cite{torresVibrationControlDeployment1996}。Vliet和Sharf则从振动模态进行考虑,建立了柔性支撑机器人的振动方程,
通过调节机器人控制器的PD增益,保证整体振动与其柔性支撑基础的主阶振动模态相匹配,通过自身结构振动消耗掉宏观的整体振动能量\cite{vlietFrequencyMatchingAlgorithm1998}。

此外,对这类基座晃动的Stewart平台,也有学者专门进行研究。Cerda在2010年针对部署在船舶上的Stewart平台,
使用比例控制和前馈控制来对支链实现直接的调控\cite{cerdasalzmannAmpelmannDevelopmentAccess2010}。
但注意到Stewart机构的惯量相对船舶是可以忽略不计的,因此其问题的耦合性比较弱。Ono等在2020、2021年针对运动基座的Stewart平台,
建立了完整的动力学模型,然后分别测试了$H_{\infty}$、integral sliding mode control(ISMC)和非线性模型预测控制方法,
验证了这三种方法对运动基座的Stwart平台控制的可靠性\cite{onoDevelopmentEquationsMotion2019}\cite{onoControlSimulationsStewart2020}
\cite{onoNonlinearModelPredictive2021}。

总的来说,针对柔性支撑的Stewart机器人,目前的控制方法主要是两种:使用PID/模糊PID的无模型方法和建立(完整)动力学模型/能量模型后的有模型方法。
针对本文的被控对象,其相对于普通的基座晃动Stewart机器人(如船舶上Stewart平台、手术并联Stewart平台)而言,由于基座是绳索牵引悬浮的,
因此Stewart机器人支链的反作用力也会影响基座的运动,这直接使得建立机构的动力学模型和辨识机构动力学参数变得复杂。此外,本模型需要辨识的参数很多,
如果使用有模型的控制方法,需要大量数据和时间进行参数辨识和分析。因此,本文采用无模型的控制思路来进行控制。

\FloatBarrier